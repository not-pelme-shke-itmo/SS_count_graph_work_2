\section{Задание 1. Дифференциальные модели первого порядка}

\textbf{Условие.}

В задачах проведите исследование:

\begin{enumerate}
    \item Составьте математическую модель задачи: введите обозначения, выпишите данные, в задаче B сделайте чертеж, составьте дифференциальное уравнениеи запишите начальные условия.
    \item Решите аналитически составленную задачу Коши.
    \item Изобразите семейство интегральных кривых и решение задачи Коши.
    \item Запишите ответ
\end{enumerate}

\begin{enumerate}[label=\Alph*.]
    \item В электрическую цепь с сопротивлением $3/2$ Ом в течение двух минут равномерно вводится напряжение (от нуля до $120$ В).
    Кроме того, автоматически вводится индуктивность, так что число, выражающее индуктивность цепи в генри, равно числу,
    выражающему ток в амперах.
    Найдите зависимость тока от времени в течение первых двух минут опыта.

    \item Найти такую кривую, проходящую через точку $(0, -2)$, чтобы угловой коэффициент касательной в любой ее точке равнялся
    ординате этой точки, увеличенной на три единицы
\end{enumerate}

\vspace{10mm}
\textbf{Решение.}

\begin{enumerate}[label=\Alph*.]
    \item It is empty but you can fill it!

    \textit{Ответ}: It is empty but you can fill it!

    \vspace{10mm}

    \item It is empty but you can fill it!

    \textit{Ответ}: It is empty but you can fill it!
\end{enumerate}

