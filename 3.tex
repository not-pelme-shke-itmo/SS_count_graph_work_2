
\section{Задание 3. ДУ второго порядка}

\textbf{Условие.}

Пружинный маятник движется по закону:

\[y^{\prime\prime} + p(t)y^\prime + q(t)y = f(t)\]

\begin{enumerate}
    \item Запишите однородное уравнение движения маятника. Выясните, почему движение описывается уравнением такого вида (каков физический смысл коэффициентов левой части уравнения).
    \item Установите характер движения (периодический, апериодический) при данных $p(t)$ и $q(t)$.
    \item Найдите ФСР ЛОДУ и убедитесь в ее линейной независимости с помощью вронскиана.
    \item Найдите общее решение ЛОДУ.
    \item Задайте начальные условия в момент $t_0 = 0$ и найдите удовлетворяющее им частное решение ЛОДУ. Изобразите закон движения в системе координат.
    \item Составьте линейное неоднородное дифференциальное уравнение (ЛНДУ) с правой частью $f(t)$. Выясните физический смысл функции $f(t)$.
    \item Найдите решение ЛНДУ, удовлетворяющее начальным условиям. Изобразите закон движения в системе координат.
    \item Сделайте вывод о влиянии на движение функции $f(t)$.
\end{enumerate}

\[p(t) = 4, q(t) = 5, f(t) = t^2 e^{2t}\]

\vspace{10mm}
\textbf{Решение.}

\begin{enumerate}
    \item $y^{\prime\prime} + 4y^\prime + 5y = 0$ \\
        Формула колебаний пружинного маятника с затуханием: \\
        $y^{\prime\prime} + \frac{c}{m}y^\prime + \frac{k}{m}y = f(t)$ \\
        Где $m$ - масса груза, $k$ - коэффициент упругости и $c$ - коэффициент затухания от скорости. \\
        Значит в идеальной системе $y^{\prime\prime} + q(t)y = t^2 e^{2t}$, ускорение зависит от координаты. \\
        В нашем случае с затуханием система должна будет стабилизироваться в нуле энергии (без движении). \\
    \item $y^{\prime\prime} + 4y^\prime + 5y = 0$ \\
        $\lambda^2 + 4\lambda + 5 = 0$ \\
        $\lambda = \pm i - 2$ \\
        $y(t) = \frac{C_1 \cos(t)}{e^{2t}} + \frac{C_2 \sin(t)}{e^{2t}}$ \\
        Движение периодическое. \\
    \item $y(t) = \frac{C_1 \cos(t)}{e^{2t}} + \frac{C_2 \sin(t)}{e^{2t}}$ \\
        ФСР: $\{\frac{\cos(t)}{e^{-2t}}, \frac{\sin(t)}{e^{-2t}}\}$ \\
        $W = \begin{vmatrix}
            \frac{\cos(t)}{e^{2t}} & \frac{\sin(t)}{e^{2t}} \\
            \frac{2\sin(t)}{e^{2t}} & \frac{-2\cos(t)}{e^{2t}}
        \end{vmatrix} = \frac{-2\cos^2(t)}{e^{4t}} + \frac{2\sin^2(t)}{e^{4t}} = \frac{2\sin^2(t)-2\cos^2(t)}{e^{4t}}$ \\
        $W \neq 0$ $\Longrightarrow$ система линейно зависима. \\
    \item Общее решение: $y(t) = \frac{C_1 \cos(t)}{e^{2t}} + \frac{C_2 \sin(t)}{e^{2t}}$ \\
    \item Пусть $y(t_0) = 1$, $y^\prime(t_0) = 0$ \\
    \item Пусть $y(t_0) = 1$, $y^\prime(t_0) = 0$ \\
        $
        \begin{cases}
            y(t_0) = \frac{C_1 \cos(t)}{e^{2t}} + \frac{C_2 \sin(t)}{e^{2t}} \\
            y^\prime(t_0) = \frac{2 C_1 \sin(t)}{e^{2t}} - \frac{2 C_2 \cos(t)}{e^{2t}}
        \end{cases}
        \begin{cases}
            1 = \frac{C_1 \cdot 1}{1} \\
            0 = -\frac{2 C_2 \cdot 1}{1}
        \end{cases}
        \begin{cases}
            1 = C_1 \\
            0 = 2C_2
        \end{cases}
        $
        Частное решение: $y = \frac{\cos(t)}{e^{2t}}$ \\
    \item $y^{\prime\prime} + 4y^\prime + 5y = t^2 e^{2t}$ \\
        $f(t)$ (в нашем случае $t^2 e^{2t}$) - действие внешних сил на систему. \\
    \item $W_1 = \begin{vmatrix}
            0  & \frac{\sin(t)}{e^{2t}} \\
            t^2 e^{2t} & \frac{-2\cos(t)}{e^{2t}}
        \end{vmatrix} = -t^2 e^{2t} \cdot \frac{\sin(t)}{e^{2t}} = -t^2 \cdot \sin(t)$\\
        $W_2 = \begin{vmatrix}
            \frac{\cos(t)}{e^{2t}} & 0\\
            \frac{2\sin(t)}{e^{2t}} &t^2 e^{2t}
        \end{vmatrix} = \frac{\cos(t)}{e^{2t}} \cdot t^2 e^{2t} = \cos(t) \cdot t^2$\\
        $C_1^\prime(t) = \frac{W_1}{W} = \frac{-t^2 \cdot \sin(t) \cdot e^{4t}}{2\sin^2(t)-2\cos^2(t)}$ \\
        $C_2^\prime(t) = \frac{W_2}{W} = \frac{t^2 \cdot \cos(t) \cdot e^{4t}}{2\sin^2(t)-2\cos^2(t)}$ \\
        $\int \frac{-t^2 \cdot \sin(t) \cdot e^{4t}}{2\sin^2(t)-2\cos^2(t)} dt = $\\
        $\int \frac{t^2 \cdot \cos(t) \cdot e^{4t}}{2\sin^2(t)-2\cos^2(t)} dt = $\\
    \item Внешняя сила позволяет стабилизировать систему в состоянии по энергии (понял по школьному курсу физики).


\end{enumerate}




